\documentclass[11pt]{article}

\newcommand{\bt}{\begin{alltt}}
\newcommand{\et}{\end{alltt}}

\usepackage[T1]{fontenc}
\usepackage[utf8]{inputenc}
\usepackage[spanish]{babel}

\usepackage[letterpaper,
            portrait,
            margin=2cm]{geometry}


\usepackage{graphicx}
\usepackage{textcomp}
\usepackage{enumerate}
\usepackage{alltt}
\usepackage{import}

\setlength{\parindent}{0pt}

\title{Inteligencia Artificial 2018-1 \\ Proyecto1: Representación del Conocimiento\\
\small{IIMAS-PCIC}}
\author{Luis Alejandro Lara Patiño\\Roberto Monroy Argumedo\\
Alejandro Ehécatl Morales Huitrón}
\date{12 de octubre de 2017}


\begin{document}

\maketitle

\section{Funcionamiento del proyecto}

Este proyecto contiene un intérprete de comandos escrito en Prolog, el cual es la base del funcionamiento del mismo.

Para iniciar la aplicación, consultar el archivo \texttt{iniciar.pl} y realizar una consulta al predicado \texttt{iniciar}, ejecutando los siguientes comandos dentro del \textit{listener} de SWI-Prolog:

\bt
    ?. [main].
    ?. iniciar.
    |:
\et

O bien, el intérprete puede ser ejecutado directamente desde una terminal, con el siguiente comando:

\bt
    swipl -f main.pl -g "iniciar, halt."
\et

Una vez iniciado el intérprete, el \textit{prompt} usual de Prolog cambiará a '\texttt{|:}'. Esto indica que ya pueden introducirse comandos.

Inicialmente, se crea una base de conocimiento vacía, la cual puede ser modificada con los comandos presentados en la siguiente sección.

\end{document}
