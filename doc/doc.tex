\documentclass[11pt]{article}

\usepackage[T1]{fontenc}
\usepackage[utf8]{inputenc}
\usepackage[spanish]{babel}
\usepackage[letterpaper,
            portrait,
            margin=2cm]{geometry}
\usepackage{graphicx}
\usepackage{textcomp}
\usepackage{enumerate}
\usepackage{enumitem}
\usepackage{alltt}
\usepackage{import}

\setlist[itemize]{noitemsep, topsep=0pt}

\newcommand{\bt}{\begin{alltt}}
\newcommand{\et}{\end{alltt}}
\newcommand{\comando}[2]{
    \textbf{#1}(#2)\\
}

\newenvironment{args}{
    Argumentos:
    \begin{itemize}
}{
    \end{itemize}
    \bigskip
}

\setlength{\parindent}{0pt}

\title{Inteligencia Artificial 2018-1 \\ Proyecto1: Representación del Conocimiento\\
\small{IIMAS-PCIC}}
\author{Luis Alejandro Lara Patiño\\Roberto Monroy Argumedo\\
Alejandro Ehécatl Morales Huitrón}
\date{12 de octubre de 2017}


\begin{document}

\maketitle

\section{Funcionamiento del proyecto}

Este proyecto contiene un intérprete de comandos escrito en Prolog, el cual es la base del funcionamiento del mismo.

Para iniciar la aplicación, consultar el archivo \texttt{iniciar.pl} y realizar una consulta al predicado \texttt{iniciar}, ejecutando los siguientes comandos dentro del \textit{listener} de SWI-Prolog:

\bt
?. [main].
?. iniciar.
|:
\et

O bien, el intérprete puede ser ejecutado directamente desde una terminal, con el siguiente comando:

\bt
swipl -f main.pl -g "iniciar, halt."
\et

Una vez iniciado el intérprete, el \textit{prompt} usual de Prolog cambiará a '\texttt{|:}'. Esto indica que ya pueden introducirse comandos.

Inicialmente, se crea una base de conocimiento vacía, la cual puede ser modificada con los comandos presentados en la siguiente sección.

\section{Referencia de comandos}

\subsection{Utilitarios}

\comando{cargar}{+Nombre}
Lee una base desde el archivo dado y la carga como la base de conocimiento actual.\\
\begin{args}
    \item Nombre. Nombre del archivo a leer. Este archivo debe existir dentro del directorio \texttt{bases}.
\end{args}

\comando{guardar}{+Nombre}
Guarda todos los contenidos de la base actual en un archivo, dentro del directorio \textit{bases}. Si el archivo no existe, se creará; si ya existe, todos sus contenidos previos se perderán.\\
\begin{args}
    \item Nombre. Nombre del archivo a escribir.
\end{args}

\textbf{ver}\\
Imprime todos los contenidos de la base actual en pantalla.

\end{document}
